\documentclass[a4paper, 11pt]{article}

\usepackage[czech]{babel}
\usepackage[utf8]{inputenc}
\usepackage[left=2cm, top=3cm, text={17cm, 24cm}]{geometry}
\usepackage{times}
\usepackage[unicode]{hyperref}
\usepackage{indentfirst}
\usepackage{graphics}
\usepackage{fancyvrb}
\usepackage{listings}
\usepackage{xcolor}
\usepackage{enumitem}
\lstset{basicstyle=\footnotesize\ttfamily,breaklines=true}
\lstset{framextopmargin=50pt,frame=none}
\hypersetup{colorlinks = true, hypertexnames = false}

\begin{document}

	\begin{titlepage}
		\begin{center}
			\LARGE\textsc{Vysoké učení technické v~Brně} \\
			\Large\textsc{Fakulta informačních technologií}\\
			\vspace{\stretch{0.382}}
			\LARGE{Network applications and management - project documentation} \\
			\vspace{0.2cm}
			\large{variant: DNS resolver} \\
			\vspace{\stretch{0.618}}
		\end{center}

		\Large{19.11.2023 \hfill Vojtěch Kališ (xkalis03)}
	\end{titlepage}

%%% TOC
	\tableofcontents

%%%
	\newpage
	\section{Introduction}
	The project's task was to create a C-based or C++-based implementation aiming to provide a 
	functional tool for Domain Name System (DNS) resolution; this specific implementation is 
	written in C. Its focus was on creating a functional resolver capable of DNS queries creation in 
	the form of UDP packets, establishing communication with given DNS server, successful sending 
	of said packet, and subsequent retrieval and processing of the response information.
	\subsection{Program arguments}
	\noindent \textbf{usage:}  dns [-r] [-x] [-6] -s server [-p port] address
	\begin{itemize}[label={}]
		\item {[-r]} = recursion desired
		\item {[-x]} = make reverse request instead of direct request
		\item {[-6]} = make request of type AAAA instead of default A
		\item -s server = IP or hostname of server to which request will be sent
		\item {[-p port]} = port number to use
		\item address = address that is the object of query(request)
	\end{itemize}
	\noindent The arguments hereby listed were all provided by the project specification. However, some specific 
	interactions weren't clearly implied and hence were up to interpretation---such as follows: 
	\begin{itemize}
		\item Reverse request (\textbf{-x}) requires \textit{server} to be an IP address, not hostname. While 
			some workaround could mayhaps be found, it was instead chosen to throw an error 
			should this scenario arise.
		\item Reverse request (\textbf{-x}) and request of type AAAA (\textbf{-6}) are incompatible, 
			as the former requires query type to be set to PTR, while the latter wishes to set it to 
			AAAA. Once again, it was chosen to throw an error in this instance as well.
	\end{itemize}
	
	\subsection{Software requirements}
		\begin{itemize}
			\item Unix-like operating system, such as Linux or macOS 
			\item A C compiler supporting the C standard libraries and functionalities
			\item Network connectivity
			\item Python version 3 for running tests
		\end{itemize}
%%%

	\newpage
	\section{Resources \& Application layout}
	Starting up (and further implementation of) the project involved a lot of web browsing and research. 
	Eventually, \href{https://www.binarytides.com/dns-query-code-in-c-with-linux-sockets/}{this basic implementation}
	\cite{article1} providing the basic functionality for DNS resolving was stumbled upon, and had been stripped and 
	used as the bare bones of the project. The \href{https://datatracker.ietf.org/doc/html/rfc1035}{RFC 1035}\cite{rfc1035} 
	was then used to restructure the DNS packet header, query and response structures, though in case of 
	header, it has been opted to use structure as shown \href{https://www.freesoft.org/CIE/RFC/2065/40.htm}{here}\cite{rfc2065} 
	instead. \href{https://datatracker.ietf.org/doc/html/rfc3596}{RFC 3596}\cite{rfc3596} then served well to implement 
	IPv6 functionality as well, especially so when reverse DNS request was concerned. DNSname and hostname conversions 
	were based on functions which can be found in \href{https://github.com/riveraj/dns-resolver/blob/master/main.c}{this implementation}
	\cite{article2}
	\subsection{Application layout (dns.c)}
	It is important to note that, while the project consists of a multitude of functions all serving their specific 
	purposes, the subject of this section is to provide overview of the main functions; that is, functions 
	which have the biggest direct contribution to the subject of the project---DNS resolving itself. These 
	functions can be found in file \textit{dns.c} underneath the \textit{INTERNAL PROGRAM FUNCTIONS} header.
	
	In case of interest in the other, auxiliary functions, the header file \textit{dns.h} provides a basic review 
	of every single function contained within the program, and an effort has also been made to provide 
	as much commentary inside the main file \textit{dns.c} as possible.

	\subsubsection{parse\_args}
	\noindent Parses command-line arguments to determine query parameters. It is here where the vast majority of invalid 
	inputs are detected and handled (program exit, corresponding errors thrown)
	\subsubsection{sock\_prep}
	\noindent Sets up sockets for communication with DNS servers, including setting timeouts and preparing IPv4 or IPv6 addresses.
	\subsubsection{dns\_pack\_prep}
	\noindent Constructs DNS packet headers with default values, enabling modifications for specific query types and flags.
	\subsubsection{dns\_qname\_insert}
	\noindent Handles the transformation of domain name into DNS name format part of DNS query preparation.
	\subsubsection{dns\_qinfo\_prep}
	\noindent Handles the query information configuration (setting Qtype and Qclass) part of DNS query preparation.
	\subsubsection{dns\_reply\_load}
	\noindent Takes DNS reponse packet and processes it by populating pre-prepared arrays for answer, authority and 
	additional records with respective data
	\subsubsection{main}
	\noindent Orchestrates the entire DNS resolution process, utilizing the above functions in their respective order to 
	handle parsing input arguments, creating a DNS packet, setting up socket and then sending prepared query 
	and receiving the answer. Lastly, it processes the response and then prints it out in required format 
	using the \textit{project\_print} function.

%%%

	\newpage
	\section{Functionality}
	The program starts its function inside the \textit{main} function, where it first calls function 
	\textit{parse\_args} to parse input arguments. Then, it creates a packet message buffer 
	large enough to contain the entire response, after which comes the task of preparing a 
	socket to connect to the desired DNS server; this icludes creating the socket and socket address 
	structures, sending them to the \textit{sock\_prep} function to set said packet up for UDP packet 
	(for DNS queries) and prepare the IPv4 or IPv6 address for the query.

	After the socket is prepared, the program creates a DNS packet header structure and points 
	it to its appropriate location within the buffer, then utilizes the \textit{dns\_pack\_prep} 
	function to initialize all header bits and sets the \textit{recursion desired} bit based on 
	if it's desired or not (meaning if '-r' argument was received or not).

	Now comes the part of preparing the dns query structure---resolving query hostname, converting it 
	into DNSname for which the \textit{dns\_qname\_insert} function is used, and inserting it into 
	the buffer right after the DNS header structure. After that, the program also creates a query 
	structure (holding the query type and class), inserts it into buffer after the query name, and fills it up.

	With the packet structure now prepared, the program attempts to send the packet to the 
	desired server and then attempts to retrieve the response (by loading it into the buffer, letting it 
	overwrite our configuration). A pointer to the portion of the packet after all previously mentioned 
	headers pushed into the buffer gets set and sent to the \textit{dns\_reply\_load} function, 
	which then starts reading the reply from then on and loading each individual record 
	into pre-prepared structure of record string arrays.

	Last but not least, the program passes all necessary data pointers and the records structure 
	into a special \textit{project\_print} function, which prints information out to stdout as per 
	the task's requirements, and then utilizes the \textit{clean\_exit}function to free any and 
	all allocated memory.


	\newpage
	\renewcommand{\refname}{Used literature}
	\bibliographystyle{plain} % We choose the "plain" reference style
	\bibliography{manual}

\end{document}